\documentclass[10pt,twocolumn,letterpaper]{article}

\renewcommand{\contentsname}{目录}
\renewcommand{\abstractname}{摘要}
\renewcommand{\refname}{参考文献}
\renewcommand{\figurename}{图}
\renewcommand{\tablename}{表}
\renewcommand{\appendixname}{附录}
\renewcommand{\listfigurename}{图}
\renewcommand{\listtablename}{表}
% \renewcommand{\lstlistlistingname}{列表目录}
% \renewcommand{\listname}{列表}

\usepackage{cvpr}
\usepackage{times}
\usepackage{epsfig}
\usepackage{graphicx}
\usepackage{amsmath}
\usepackage{amssymb}  
\usepackage{amsthm}  
% \usepackage{multicolumn}
\usepackage{multirow}
  
\usepackage{algorithm}  
\usepackage{algorithmic}  

\usepackage{listings}  

% \lstset{ %
% language=python,                % the language of the code
% basicstyle=\footnotesize,           % the size of the fonts that are used for the code
% numbers=left,                   % where to put the line-numbers
% numberstyle=\tiny\color{gray},  % the style that is used for the line-numbers
% stepnumber=2,                   % the step between two line-numbers. If it's 1, each line 
%                                 % will be numbered
% numbersep=5pt,                  % how far the line-numbers are from the code
% backgroundcolor=\color{white},      % choose the background color. You must add \usepackage{color}
% showspaces=false,               % show spaces adding particular underscores
% showstringspaces=false,         % underline spaces within strings
% showtabs=false,                 % show tabs within strings adding particular underscores
% frame=single,                   % adds a frame around the code
% rulecolor=\color{black},        % if not set, the frame-color may be changed on line-breaks within not-black text (e.g. commens (green here))
% tabsize=2,                      % sets default tabsize to 2 spaces
% captionpos=b,                   % sets the caption-position to bottom
% breaklines=true,                % sets automatic line breaking
% breakatwhitespace=false,        % sets if automatic breaks should only happen at whitespace
% title=\lstname,                 % show the filename of files included with \lstinputlisting;
%                                 % also try caption instead of title
% keywordstyle=\color{blue},          % keyword style
% commentstyle=\color{dkgreen},       % comment style
% stringstyle=\color{mauve},         % string literal style
% escapeinside={\%*}{*)},            % if you want to add LaTeX within your code
% morekeywords={*,...}               % if you want to add more keywords to the set

% Include other packages here, before hyperref.

% If you comment hyperref and then uncomment it, you should delete
% egpaper.aux before re-running latex.  (Or just hit 'q' on the first latex
% run, let it finish, and you should be clear).
\usepackage[breaklinks=true,bookmarks=false]{hyperref}

\usepackage[utf8]{inputenc} % allow utf-8 input
\usepackage[T1]{fontenc}    % use 8-bit T1 fonts
\usepackage{hyperref}       % hyperlinks
\usepackage{url}            % simple URL typesetting
\usepackage{booktabs}       % professional-quality tables
\usepackage{amsfonts}       % blackboard math symbols
\usepackage{nicefrac}       % compact symbols for 1/2, etc.
\usepackage{microtype}      % microtypography

\usepackage{url}  
\usepackage{caption}
% \usepackage{subfigure*}
% \usepackage{paisubfigure*}
\usepackage{graphicx}
\usepackage{booktabs}
% \usepackage{amsmath}

\makeatletter
\let\@afterindentfalse\@afterindenttrue
\@afterindenttrue
\makeatother
\setlength{\parindent}{2em}

\usepackage{amsmath,amssymb}
\DeclareMathOperator{\E}{\mathbb{E}}  

\renewcommand{\baselinestretch}{1.2}
\usepackage{xeCJK}
\usepackage{fontspec} 
\setCJKmainfont{SimSun} %或\setCJKmainfont{KaiTi}
\setCJKmonofont{SimSun}
\setmainfont{Times New Roman}

\cvprfinalcopy % *** Uncomment this line for the final submission

% \def\cvprPaperID{****} % *** Enter the CVPR Paper ID here
% \def\httilde{\mbox{\tt\raisebox{-.5ex}{\symbol{126}}}}

% Pages are numbered in submission mode, and unnumbered in camera-ready
%\ifcvprfinal\pagestyle{empty}\fi
% \setcounter{page}{1}
\begin{document}
% \newtheorom{algorithm}{算法}
\title{多人博弈``十点半”游戏的若干问题分析}

\author{易凯\thanks{The author received his B.Eng with honor from Department of Software Engineering, Xi'an Jiaotong University in June 2019. His current research interests include cognition-based artificial intelligence, machine learning, computer vision and conputational psychology. His homepage is kaiyi.me. Now, he is planning to pursue PhD studies and internships.}
% Institution1 address\\
% {\tt\small {yikai2015, yushuanghe1997, sherlockholmes, qyl916}@stu.xjtu.edu.cn}
% For a paper whose authors are all at the same institution,
% omit the following lines up until the closing ``}''.
% Additional authors and addresses can be added with ``\and'',
% just like the second author.
% To save space, use either the email address or home page, not both
}
% \providecommand{\keywords}[1]{\textbf{\textit{Index terms---}} #1}
% \providecommand{\keywords}[1]{\textbf{\textit{关键词---}} #1}

\maketitle

\section{游戏简介}
``十点半“游戏是多人博弈游戏,其由1个庄家和N个闲家构成。基本的游戏规则以及流程如下:

\begin{algorithm}[!htp]
\caption{``十点半"游戏规则}
\label{alg:A}
\begin{algorithmic}
\STATE \textbf{步骤一}:庄家和N个闲家依次发一张牌(默认每次游戏结束重新洗牌)。所有数字牌记为实际点数,花牌记为半点(包括大小王)。
\STATE \textbf{步骤二}:庄家决定轮。庄家可以自由选择补牌(持牌数上限为五张),庄家若总点数为十点半,则庄家赢。若总牌数为五张,若总点数小于等于十点半,则庄家赢(若等于十点半则闲家输双倍底金);若总点数大于十点半,则闲家赢,获得资本与底金同等数额。上述所有情况游戏结束。
\STATE \textbf{步骤三}:闲家$i\in [1, N]$ 补牌,若闲家 $i$ 点数超过十点半,则闲家$i$输;若闲家 $i$ 点数等于十点半或者持牌数等于五且总点数小于等于十点半,则闲家 $i$ 赢,且庄家需付给 $i$ 底金的两倍(若持牌数等于五且总点数等于十点半,则庄家需付给 $i$ 底金的四倍)。
\STATE \textbf{步骤四}:重复步骤三,直到所有闲家完成补牌。庄家与所有剩余闲家比点数,若庄家点数小于闲家$i$,则闲家$i$胜,否则,庄家胜。胜负额度均与闲家 $i$ 底金相同。游戏结束。
\end{algorithmic}
\end{algorithm}

\section{双人博弈基本分析}
双人博弈指的是一庄一闲的基本情况,主要分析两点,庄家和闲家获得五张牌十点半的概率以及庄闲的数学期望。
\subsection{双人博弈庄闲五张牌十点半的概率分析}
由于相关问题的概率分析较为复杂,因此通过数学模拟的手段进行分析。

经过50,000次迭代,庄家五张牌十点半的概率为0.012308, 闲家五张牌十点半的概率为0.010168。
\subsection{双人博弈庄闲获胜期望分析}
此处做出两个假设:

\begin{itemize}
   \item 庄家补牌或停牌决策点为7,任一闲家决策点为7.5;
   \item 每次结束一局均重新洗牌
\end{itemize}

通过演算,庄家的胜率为0.6451,远大于0.5。此外,庄家的数学期望为 0.2074,闲家的数学期望为 -0.2074. 

\section{多人博弈庄闲获胜期望分析}
多人博弈庄闲获胜期望分析如 表 \ref{lab:1},其中每组数据均经过了50,000 次数学模拟,使用 Python3 实现。

\begin{table}
   \centering
   \caption{多人博弈庄闲获胜期望分析(仅保留四位小数, N表示闲家数)}
   \label{lab:1}
   \small
   \begin{tabular}{c|c|c|c|c|c|c}
      \hline
      N & Dealer & Player 1 & Player 2 & Player 3 & Player 4 & Player 5\\
      \hline
      1 & 0.2074 & -0.2074 & - & - & - & - \\
      2 & 0.5028 & -0.2510 & -0.2517& - & - & - \\
      3 & 0.8374 & -0.2803 & -0.2765 & -0.2806& - & - \\
      4 & 1.1851 & -0.2842 & -0.3016 & -0.2997 & -0.2996& - \\
      5 & 1.5206 & -0.2959 & -0.3031 & -0.3024 & -0.3047 & -0.3146\\
      \hline
   \end{tabular}
\end{table}

\section{结论}
\begin{itemize}
   \item 通过表 \ref{lab:1} 不同闲家数时各个闲家的数学期望基本相同来看,闲家获胜的数学期望与发排顺序无关;
   \item 通过表 \ref{lab:1} 庄闲数学期望的对比来看,庄家的数学期望始终为正,闲家的数学期望始终为负,且闲家数量越多,庄家的数学期望越高。
\end{itemize}

% {\small    
% \bibliographystyle{unsrt}
% \bibliography{egbib}
% }

\end{document}
